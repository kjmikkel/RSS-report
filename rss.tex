% -*- coding: utf-8 -*-

\section{Rectangular Swept Spheres}
\label{rss}

\subsection{Introduction}
In this section I will go into greater detail about the RSS, how it is defined and represented, as well as the literature that I have used in this report.

\subsection{Rectangular Swept Sphere}
The Rectangular Swept Sphere (RSS) is a 3D figure. It is generated by taking a sphere with a non-zero radius and sweeping it over a rectangle in 3 dimensional space. This makes it reaking it resemble a rectangular rounded pillow \Sfixme{is this a reasonable description}. An illustration of a RSS can be found on p. 10 of \cite{Larsen99fastproximity}.\Sfixme{Should I even refer to this - am I just duplicating what I have in the introduction} 

\subsection{Representation of RSS}
I have chosen to represent the RSS as a rectangle in 3D together with a radius, as described in \cite{larsen00fast}, \cite{Larsen99fastproximity} and \cite{237244}. See section \ref{rectangle3d} page \pageref{rectangle3d} for  implementation details. I felt this was best way to represent it, as it adequately described the RSS, took up a minimum of space, contains the vectors, and follows the representation found in \cite{237244}, which is useful, as I use the axis-separation test found in that article.

\subsection{Overlap of 2 RSS'}
\Sfixme{think about whether this is important enough to warrent a section and if so mention how we find out when 2 RSS' are intersecting}

\subsection{Literature used in this report}
For this project I have used the following literature:
\begin{description}
\item[\cite{larsen00fast}] Gives an introduction to some of the problems with making Distance Queries between RSSs. The article refers \cite{Larsen99fastproximity} (by the same authors) which contains more explicit information on how to calculate the distance. 
\item[\cite{Lotan03algorithmand}] Gives a general introduction to Monte Carlo simulation of Proteins, as well as a comparison of several data-structures and algorithms for these, for protein folding \Sfixme{Is it correct? It can be written better}
\item[\cite{Larsen99fastproximity}] A more in-depth description of how the distance queries for the RSSs can be preformed. However, it leaves the most difficult cases to \cite{237244}.
\item[\cite{237244}] A description the algorithm for the most difficult case of RSS distance query. Although the algorithm as described is for Oriented Bounding Boxes (OBB).
\end{description}
