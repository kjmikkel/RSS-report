% -*- coding: utf-8 -*-

\section{Results}
\label{results}
\subsection{Introduction}
In this section I will discuss the results of the project.

\subsection{Implemented features}
I have successfully implemented the RSS data-structure as well as its dependent class, a rectangle in 3D. 

Furthermore I have implemented

\subsection{Comparisons}

In order to gauge whether my implementation was successful or not, I have tried to compare it with several other implementations - namely LSS' and OBBs. The parameters for comparison have been their volume and the runtime required to check whether 2 BV are separate or not.

\subsubsection{Generation of tests results}
All of the test results have been generated by generating 2000 random sets of points (25 points in each set), testing them both for volume and overlap. The value for the overlap is an average, while the run time is the total wall clock time expended on the operation.

All of these tests have been executed on a [my own computer specs].

\subsubsection{Volume}


\subsubsection{Overlap detection}

\begin{figure}
\caption{\label{overlap-OBB_RSS_RSS2_LSS}Comparison of the time spent for the different algorithms}
\end{figure}

As it is clear from the comparisons in table [reference], the current implementation of the current version of the RSS overlap algorithm is clearly slower than both the one implemented for the OBBs, and the LSS. 

This can partly be explained by the fact that due to time constraints little optimizations have been performed for the overlap method for RSS. The extra time required does however not seem insurmountable, and future optimizations are likely to bring the time required for RSS' to check for overlap closer, and perhaps even under, the time required for the OBB's.

\subsubsection{Problems with the current tests}
\label{test_problems}
Due to time constraints I have not been able to implement the RSS into the BVH, and hence all tests have been done on random generated points. It is therefore likely that in most of the cases there will not be an overlap, and therefore overlap tests that are fast when 2 BV's are disjoint will perform better than algorithms that are fast when the 2 BV's are not disjoint (such as the overlap check for the RSS). Since overlaps are more likely to occur in practice \Sfixme{Can I justify this?}, this gives the algorithm I have implemented a clear disadvantage compared to OBB's.

It is worth mentioning, that as implemented, OBB's and RSS' both uses the Axis separation test, but as mentioned in \ref{intersection}, RSS first checks the minimum distance between the two RSS' is less than their combined radii. So it is clear that in the worst case, the RSS overlap check has to do more work than the worst case overlap test for OBB's, namely the minimum distance check. It is therefore clear that if the overall time spent on the RSS overlap test has to be less than the time spent on the OBB overlap test, then this advantage has to come from a tighter volume fit, in favour of the RSS. 

Furthermore, as described in \ref{implementation_axis_sep}, given 2 RSS' A and B, I have not made the axis go through A (which in practice would mean that I would have had to find the transformation of A so that it became axis-parallel, and then transformed B so with the same transformation) \Sfixme{Better explanation or remove entirely?}. Since all OBB's in ProGAL are axis oriented, this is trivially true for the axis-separation test, with all the possible optimizations that enables.
