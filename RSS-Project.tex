% -*- coding: utf-8 -*-
\documentclass[11pt, oneside, a4paper]{article}
\usepackage{mikkel}
\usepackage[ruled,vlined]{algorithm2e}
\usepackage{algorithmic}
% A command to globally silence fixmes
\newcommand{\Sfixme}[1]{}
%\newcommand{\Sfixme}[1]{\fixme{#1}}
\title{Rectangular Swept Spheres}
\author{Mikkel Kjær Jensen}
\date{\today}
\begin{document}
\pagestyle{headings}
\maketitle

\abstract{
  Protein folding is a important method for finding new and better drugs. In order to ensure that the proteins can exist in reality, each fold (in the domain known as conformation) has to be checked, to see if it does not make the protein overlap itself. One method is using a Bounding Volume Hierarchies, which can be used to efficiently compare Bounding Volumes. I have in this project implemented the Rectangular Swept Sphere Bounding Volume (which is generated by sweeping the center point of a sphere through a rectangle in 3d), and attempted to implement the overlap detection found in ``Fast Distance Queries with Rectangular Swept Sphere Volumes'' by Larsen et al. The current implementation has been tested against both Oriented Bounding Boxes and Linear Swept Sphere, and has been shown to be inferiour to both. However, it is my estimate that this is because the current implementaion suffers from a misunderstanding of the method described in the above article, and that if implemented correctly it could prove more efficient than Oriented Bounding Boxes..
}

\clearpage
\listoffixmes
\tableofcontents

% -*- coding: utf-8 -*-

\section{Introduction}
\label{introduction}

The area of protein folding (and from this the creation of newer and better drugs) with the aid of computers, has since the 60's been an area of great research and experimentation. 


The influence of computers in modern medicine is becoming increasingly important. In order to find new and better drugs scientists have turned to computer in order to simulate human biology.

An often used method is protein folding, where different foldings of proteins are tested for interesting properties. Since proteins cannot physically overlap with another part, the program has to check whether there is a collision or not. Since this test has to be done often, it is critical that the overlap tests can be done fast. 

One method for detecting collisions is Bounding Volumes Hierarchy (BVH), using a Bounding Volume (BV), where each node in the BVH is enveloped by a BV that covers the all children of the nodes (with the leafs of the node being parts of the protein itself). If a BV does not intersect any other BV, then it is clear that the attempted fold is legal, and it can take place. If there is a collision between two or more BVs, then further checks are needed, in order to check whether there truly is an overlap.

One of possible BV is the Rectangular Swept Sphere (RSS), which is generated by sweeping a sphere (with a positive radius) over a rectangle in 3D space. The resulting volume looks something like a pillow - see figure \ref{rss-example} on page \pageref{rss-example}. 

I will in this project attempt to implement a reasonably effective RSS BV, with the ability to test for overlap detection, the ability to create a new RSS from a set of points, and the ability to crate a new RSS that contains all the points of 2 (possibly distinct) RSS'.

\subsection{Scope and Limitations}
\label{scope}

\subsection{Expectations to the reader}
\hide{
I expect the reader to have a rudimentary knowledge of the problems encountered with protein folding, as well as familiarity with the concepts such as Bounding Volumes and Swept Spheres.} I expect the reader to have a good grasp of computational geometry and, vector manipulations and algorithms in general.

\subsection{Terminology}

\subsection{Guide to this report}
\begin{description}
\item[Section \ref{introduction}] The introduction to the report, which will introduce the subject and explain my goals
\item[Section \ref{rss}] A description of the Rectangular Swept Spheres - how they are defined in theory, and how I have chosen to represent them. This section will also contain information about some of the literature that has worked with RSS.
\item[Section \ref{algorithms}] The algorithms that work on the RSSs, and an analysis of their run time 
\item[Section \ref{implementation}] Notes on the implementation of the algorithms from the previous section and the RSS itself in ProGAL.
\item[Section \ref{results}] A discussion about the results of the implementation together with a comparison with Oriented Bounding Boxes. 
\item[Section \ref{conclusion}] The conclusion of the report, where I will sum up the findings 
\end{description}

\subsection{Literature used in this report}
For this project I have used the following literature: \Sfixme{find out whether it should be here, or in section 2}
\begin{description}
\item[\cite{larsen00fast}] Gives an introduction to some of the problems with making Distance Queries between RSSs. The article refers \cite{Larsen99fastproximity} (by the same authors) which contains more explicit information on how to calculate the distance. 
\item[\cite{Lotan03algorithmand}] Gives a general introduction to Monte Carlo simulation of Proteins, as well as a comparison of several data-structures and algorithms for these, for protein folding. None of the material in this article is used directly, but rather as background material.
\item[\cite{Larsen99fastproximity}] A more in-depth description of how the distance queries for the RSSs can be preformed. However, it leaves the most difficult cases to \cite{237244}.
\item[\cite{237244}] A description the algorithm for the most difficult case of RSS distance query. Although the algorithm as described is for Oriented Bounding Boxes (OBB).
\end{description}

% -*- coding: utf-8 -*-

\section{Rectangular Swept Spheres}
\label{rss}


\subsection{Representation of RSS}
I have chosen to represent the RSS as a rectangle in 3D together with a radius, as described in \cite{larsen00fast}, \cite{Larsen99fastproximity} and \cite{237244}. The rectangle is made up from a 3d point which is its center, 2 unit vectors in 3d which describes the slope \fixme{is this a correct term} of the rectangle, and the length of the of the original 3d vectors.

I have choosen this implementation, since \cite{327244} uses it, and I intend to use the overlap algorithm found in \cite{237244}.

% -*- coding: utf-8 -*-

\section{Algorithms}
\label{algorithms}
\subsection{Introduction}
In this section I will go into high-level detail about the principal problems that are to be solved in this project, as well as the algorithms that I have used to solve them. With each algorithm I will include an explanation of their use, an analysis of the algorithms run-time, as well as a discussion of possible alternatives. 

\subsection{Intersection}
In order to be able to use the Rectangular Swept Sphere (RSS) in ProGAL framework \Sfixme{is that what it is called}, it is paramount that I must find a fast way to detect when 2 RSS' are intersecting, so that it can be decided whether a fold of the protein should go ahead or not. In the following I will describe the methods I have used to check whether or not two RSS' are intersecting.

\subsubsection{Approach}
In the algorithm I have chosen to implement, I will prefer to report a possible intersection where there might not be one, instead of failing to report a possible overlap. While this may give a performance penalty, since the system might have to perform more checks, it is critically important that illegal configurations of the proteins are not accepted. 

\subsubsection{Minimum distance}
In the following let minDist(rec(A), rec(B)) be the minimum distance between the rectangles of RSS A and B, radius(A) is the radius of A.\\

One approach to check if 2 Rectangular Swept Spheres, A and B, overlap, is to check the minimum distance between their rectangles. It is clear that if minDist(rec(A), rec(B)) $<=$ Radius(A) + radius(B) \Sfixme{implement in code and show illustation}, then A and B overlaps. This is the approach found in \cite{Larsen99fastproximity}.

The problem then becomes one of finding the 2 closest points, and calculating the distance.
Given that the rectangles do not overlap, then the possible configurations of the closest points are, according to section 4.2.1 of \cite{Larsen99fastproximity}:
\begin{enumerate}
\item Both of the points lie on an edge
\item One of the points lie in the interior of one of the rectangles, while the other either lies on the other rectangles edge or interior.
\end{enumerate}

\subsection{The points lie on 2 edges}
\label{minimumDistance}
Since we are only interested in the smallest minimum distance between the edges, it is clear that we only have calculate the minimum distance between the edges that are closest to each other. The approach in my implementation is the one described in \cite{larsen00fast} and \cite{Larsen99fastproximity}, where the authors exploit the properties of Voronoi diagrams\footnote{For a formal description of Voronoi diagrams see \cite{compgeom:2008} Chapter 7} in order to quickly decide which pair of edges are possible candidate for the smallest distance. For a more detailed justification of this, see \cite{larsen00fast} section 4.2 and \cite{Larsen99fastproximity} section 4.3.1 - particularly Lemma 1.

Fortunately one does not need to calculate the Voronoi diagram in order to detect which Voronoi cell an edge lies in. Let E be the edge whoose closest edges we wish to find, \textbf{n} be the perpendicular vector to E, then the face defined by -\textbf{n} and a point on E will define the half-plane D. All edges that lies, wholey or partly, lies within D are candiates for the closest edge \cite{larsen00fast} \Sfixme{Check that I do not perform plagerism}. However, this has the minor disadvantage that an edge might lie in multiple half-planes, but even in the worst case this would mean that 8 checks would have had to be made (2 checks for each edge), which clearly is better than 16 checks.\Sfixme{perhaps there should be an image illustrating this - yes there proberly should}

The problem then becomes one of finding the minimum distance between 2 line-segments in 3 dimensions, and then returning the smallest of these values. If no possible candidates are found, then infinity is returned.

If RSS' overlap, the closets point the points with the minimum distance to the  , or the minimum distance between the 2 RSS is greater than the maximum radius, then it becomes necessary to run the separating axis test, which is described below\Sfixme{This can be formulated better}.

\subsection{Separating axis test}
\label{sepAxis}
In order to take care of the cases where 2 rectangles either overlap or where the closest point lies inside the interior of the rectangles, I have to do an Axis-separation check, as described in \cite{237244}. The gist of the Axis-separation test is ``that two disjoint convex polytopes in 3-space can always be separated by a plane which is parallel to a face of either polytope, or parallel to an edge from each polytope'' (\cite{237244}, section 5, page. 8). If one or more Axis-separation shows that there might be an overlap, then the system will report that further test are needed.

The Axis-separation test described in \cite{237244} was designed to work on Oriented Bounding Boxes (OBB's), and takes care of rectangle to rectangle axis test as a degenerate case (\cite{237244}, section 5), which is faster. 

However, since each RSS has a radius, it became a question during implementation whether the system should assume that the two objects it was comparing should be treated OBB's (with a length, a width and a height), or if they were rectangles (with only a length and a width). In the end I choose to implement the axis test to take care of OBB's, and treat the overlap for rectangles as a special case (where the height of the two rectangles are 0).

\subsubsection{Order of Minimum distance and Axis separating test}
\label{minAxisOrder}
A pertinent question might be whatever the order of the tests are optimal, or whether it might be better just to run the axis separation test.

In order to have this discussion make any sense, it is important to understand the purpose and the exiting condition for both methods:

\begin{description}
\item[Minimum Distance:] The minimum distance algorithm tries to find all the minimum distances it can find. 
Local best case: None of the edges lies in the half-planes that indicates the  Voronoi cells\Sfixme{other description?}, and it terminates, having found the minimum distance between no edges. The exiting condition of the algorithm is when it either has checked all Voronoi cells \Sfixme{description again?} and calculated all the minimum distances, or when it finds a minimum distance that is smaller than the combined radius of the two RSS' \Sfixme{implement this in the code}. If the algorithm returns with a positive, then we know that the 2 RSS' intersect, but if it returns with a negative, then we do not know whether the 2 RSS' overlap or not. The algorithm favours the cases where the 2 RSS' might overlap.  

\item[Axis Separating:] The axis separation algorithm tries to find an axis that separates the 2 RSS'. In the best case, the first axis will show that the two RSS' are disjoint, and the algorithm terminates. In the worst case the 2 RSS' are not independent, and all 15 (6 axis tests for 3 faces from each box in the RSS \Sfixme{explain definition of RSS back in appropriate section} + 9 axises that are created by a combination of the 3 face vectors from each box\Sfixme{reformulate!}) test have to be made. The algorithm favours the cases where the 2 RSS' are disjoint.
\end{description}

From this it is clear that the order of the algorithm we choose will be influenced by which cases we believe to be the most common, as well as which cases we think of as the most likely. If we think that the 2 RSS' will often be above each other, or that the RSS' often will be separated, then it will be better to only run the Axis-separation test. If on the other hand we believe that RSS' won't be placed above each other, and that they often intersect, then the minimum distance algorithm, followed by the Axis-separation test is preferred.

Since in the context of folding proteins are interested in being told that a fold is illegal as soon as possible (so that the next fold can be tested), I think that first running the Minimum distance algorithm, and then the Axis-separation test would improve the run-time in most instances, compared to if we only ran the Axis-separation test.

\subsection{Heuristic for creating RSS for multiple points}
In order for the RSS to be a useful data-structure, it is clear that, given a set of points, we must be able to construct a RSS that contains all of the points. If this was not the case, then we would be unable to use RSS in a Bounding Volume Hierarchy\Sfixme{reference to text and page}.

In order to create the RSS from multiple points in 3d, I first find the covariance matrix\Sfixme{Should there be something about what a covariance matrix is?}, and then project all the points down into the 2 dimensional xy-plane. From these 2-dimensional points I then create a 2d rectangle in this plane, and gets the 3 corner points which defines it. These 3 points are then projected up into the plane created by the covariance matrix, and from these 3d points a 3d rectangle is constructed. I then find the maximum distance from the 3d-rectangle to the 3d points, which I set as the radius. I then use the 3d rectangle and the Radius to create the RSS.

\begin{algorithm}[H]
  \caption{CreateRSSContainingPoints}
  \SetKwData{covar}{CovarianceMatrix}
  \SetKwData{twodeeRec}{2dRec} \SetKwData{twodeeP}{2dPointset}
  \SetKwData{cornTwo}{2dRectangleCornerPoints}
  \SetKwData{cornThree}{3dRectangleCornerPoints}
  \SetKwData{threedeeRec}{3dRec}
  \SetKwData{return}{return}
  \SetKwData{p}{p}
  \SetKwInOut{Input}{input} \SetKwInOut{Output}{output}
  \dontprintsemicolon
  \Input{A set P of n points in 3 dimensions}
  \Output{A RSS that contains all points in P}
  Initialize \twodeeP, \cornThree \;
  \covar $\gets$ makeCovarainceMatrix(P)\;
  \ForEach{\p $\in P$}{
    Add projectTo2d(\p) to \twodeeP \;
  }
  \twodeeRec $\gets$ make2dRectangle(\twodeeP) \;
  \cornTwo $\gets$ getCornerPoints(\twodeeRec) \;
  \ForEach{\p $\in$ \cornTwo}{
    Add projectToCovariance(\covar) to \cornThree \;
  }
  \threedeeRec $\gets$ make3dRec(\cornThree) \;
  maxDistance $\gets 0$ \;
  \ForEach{\p $\in P$}{
    maxDistance $\gets$ max(maxDistance, distanceToRec(\p, \threedeeRec))
  }
  \return makeRSS(\threedeeRec, maxDistance)
\end{algorithm}

\subsubsection{Asymptotic Runtime}
From the algorithm it is clear that everything besides the creation of the covariance matrix runs in $O(n)$ time. Since the covariance matrix also is created in $O(n)$ time, it is clear that the entire algorithm runs in $O(n)$ time.

\subsection{Algorithm to create a RSS containing 2 RSS'}
In order to combine  2 RSS', rss1 and rss2, I first find the 6 points that define the 2 RSS'. I then create a new RSS, rss3, which contains all 6 points, using the algorithm above. Once that is done, I add  max(radius(rss1), radius(rss2)) to the radius of rss3, ensuring that all points contained in both rss1 and rss2 are contained in RSS3. This process can easily be extended to multiple RSS', by first collecting all the points, 

\begin{algorithm}[H]
  \caption{CombinedRSS}
  \SetKwData{points}{pointsSet}
  \SetKwData{crss}{CombinedRSS}
  \SetKwData{return}{return}
  \SetKwInOut{Input}{input} \SetKwInOut{Output}{output}
  \dontprintsemicolon
  \Input{2 RSS' rss1 and rss2}
  \Output{A RSS that contains both RSS'}
  Initialize \points \;
  Add cornerPoints(rss1) and cornerPoints(rss2) to \points \;
  maxRadius $\gets$ max(radius(rss1), radius(rss2)) \;
  \crss $\gets$ CreateRSSContainingPoints(\points) \;
  \crss $\gets$ addRadius(\crss, maxRadius) \;
  \return \crss
\end{algorithm}

\subsubsection{Runtime}
It is clear that the algorithm must run in $O(1)$ since \texttt{CreateRSS} runs in $O(n)$, and it will always be called with $O(1)$ points.

%\subsection{Non-axis aligned rectangle}

% -*- coding: utf-8 -*-

\section{Implementation}
\label{implementation}

\subsection{Introduction}
In this section I will give an overview of the implementation details of this projects. This will include choice of programming language, libraries that I have found useful, as well as an overview of the classes and files that I have added to the framework.

\subsection{Choice of programming language}
Since the goal of this project is to implement the Rectangular Swept Sphere into the ProGAL Java framework there has never been any doubt that as much as possible of this project should be implemented in Java - only using other languages if there was already existing, efficient, code, that implemented critical features.

\subsection{Use of Framework}
Since this project is about implementing RSS in the ProGAL framework, I will make heavy use of the features already existing in the framework. This is done both to save time, and to avoid code duplication, but also because I assume that the implemented features would be more efficient than what I would be able to implement.

\subsection{Classes implemented}
In this project I have implemented 2 classes:
\begin{itemize}
\item Rectangle3d
\item RSS3d
\end{itemize}

which primarily uses these custom classes form ProGAL:

\begin{itemize}
\item[Point3d:] A point in 3D
\item[Vector3d:] A vector in 3D
\item[Volume3d:] An interface that requires the implementation of an \texttt{overlaps} and \texttt{volume}, which respectively tests whether 2 Voule3d overlaps, and returns the volume of the Volume3d object as a double.
\end{itemize}

\subsubsection{Rectangle3d}
\label{rectangle3d}
The Rectangle3d class is used to represent a rectangle in 3 dimensional space. To represent the rectangle itself I use one center point (Point3d), 2 unit vectors (Vector3d) which describe the plane that the rectangle lies in, and 2 doubles that contains the length of the vectors. In order to comply with the other structures, the rectangle3d class implements the interface Volume3d.

It is also in this class that I have implemented both the Minimum distance algorithm and Axis-separation algorithm. My justification for implementing it in rectangle instead of the RSS3d class is that both algorithms works on rectangles/rectangular boxes, and not only on Rectangular Swept Spheres or any other data-structures.

\subsubsection{RSS3d}
\label{RSS3d}
The RSS3d class is used to represent a Rectangular Swept Sphere in 3 dimensional space. The RSS is represented by a radius which is given as a double, and the rectangle is a Rectangle3d, as descried above. In order to comply with the other structures, the rectangle3d class implements the interface Volume3d.

In order to check whether 2 RSS' intersect, one of the RSS' overlap method is called with the other RSS as argument.

\fixme{write this part when it has been decided which method you will use to create the RSS}
In this project I have tried 2 different methods for constructing the RSS from a set of points:

The first method I tried took the created an axis-oriented rectangle in 2d, which was then projected into 3d. This implementation is clearly inferior to one that uses a non axis-oriented rectangle, since the latter solution would create a more compact bounding volumes - which presumably would reduce the number of intersection tests. Due to time constraints I have not tried to implement this solution, though I belive it to be worth investigating. 

% -*- coding: utf-8 -*-

\section{Results}
\label{results}
\subsection{Introduction}
In this section I will discuss the results of the project.

\subsection{Implemented features}
I have successfully implemented the RSS data-structure as well as its dependent class, a rectangle in 3D. 

Furthermore I have implemented

\subsection{Comparisons}

In order to gauge whether my implementation was successful or not, I have tried to compare it with several other implementations - namely LSS' and OBBs. The parameters for comparison have been their volume and the runtime required to check whether 2 BV are separate or not.

\subsubsection{Generation of tests results}
All of the test results have been generated by generating 2000 random sets of points (25 points in each set), testing them both for volume and overlap. The value for the overlap is an average, while the run time is the total wall clock time expended on the operation.

All of these tests have been executed on a [my own computer specs].

\subsubsection{Volume}


\subsubsection{Overlap detection}

\begin{figure}
\caption{\label{overlap-OBB_RSS_RSS2_LSS}Comparison of the time spent for the different algorithms}
\end{figure}

As it is clear from the comparisons in table [reference], the current implementation of the current version of the RSS overlap algorithm is clearly slower than both the one implemented for the OBBs, and the LSS. 

This can partly be explained by the fact that due to time constraints little optimizations have been performed for the overlap method for RSS. The extra time required does however not seem insurmountable, and future optimizations are likely to bring the time required for RSS' to check for overlap closer, and perhaps even under, the time required for the OBB's.

\subsubsection{Problems with the current tests}
\label{test_problems}
Due to time constraints I have not been able to implement the RSS into the BVH, and hence all tests have been done on random generated points. It is therefore likely that in most of the cases there will not be an overlap, and therefore overlap tests that are fast when 2 BV's are disjoint will perform better than algorithms that are fast when the 2 BV's are not disjoint (such as the overlap check for the RSS). Since overlaps are more likely to occur in practice \Sfixme{Can I justify this?}, this gives the algorithm I have implemented a clear disadvantage compared to OBB's.

It is worth mentioning, that as implemented, OBB's and RSS' both uses the Axis separation test, but as mentioned in \ref{intersection}, RSS first checks the minimum distance between the two RSS' is less than their combined radii. So it is clear that in the worst case, the RSS overlap check has to do more work than the worst case overlap test for OBB's, namely the minimum distance check. It is therefore clear that if the overall time spent on the RSS overlap test has to be less than the time spent on the OBB overlap test, then this advantage has to come from a tighter volume fit, in favour of the RSS. 

Furthermore, as described in \ref{implementation_axis_sep}, given 2 RSS' A and B, I have not made the axis go through A (which in practice would mean that I would have had to find the transformation of A so that it became axis-parallel, and then transformed B so with the same transformation) \Sfixme{Better explanation or remove entirely?}. Since all OBB's in ProGAL are axis oriented, this is trivially true for the axis-separation test, with all the possible optimizations that enables.

% -*- coding: utf-8 -*-

\section{Conclusion}
\label{conclusion}
\subsection{Introduction}
In this section I will attempt 


\subsection{Future work}
Another Swept Sphere variation it might be worth looking into is Triangular Swept Spheres (TSS), where instead of a rectangle or a line (as is the case in RSS and LSS) a triangle is used as the surface for the sphere to swept through.


\bibliography{RSSbib}{}
\bibliographystyle{plain}
\end{document}
