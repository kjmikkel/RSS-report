% -*- coding: utf-8 -*-
\documentclass[11pt, oneside, a4paper]{article}
\usepackage{mikkel}
\usepackage[ruled,vlined]{algorithm2e}
\usepackage{algorithmic}
% A command to globally silence fixmes
\newcommand{\Sfixme}[1]{}
%\newcommand{\Sfixme}[1]{\fixme{#1}}
\title{Rectangular Swept Spheres}
\author{Mikkel Kjær Jensen}
\date{\today}
\begin{document}
\pagestyle{headings}
\maketitle

\abstract{
  Protein folding is a important method for finding new and better drugs. In order to ensure that the proteins can exist in reality, each fold (in the domain known as conformation) has to be checked, to see if it does not make the protein overlap itself. One method is using a Bounding Volume Hierarchies, which can be used to efficiently compare Bounding Volumes. I have in this project implemented the Rectangular Swept Sphere Bounding Volume (which is generated by sweeping the center point of a sphere through a rectangle in 3d), and attempted to implement the overlap detection found in ``Fast Distance Queries with Rectangular Swept Sphere Volumes'' by Larsen et al. The current implementation has been tested against both Oriented Bounding Boxes and Linear Swept Sphere, and has been shown to be inferiour to both. However, it is my estimate that this is because the current implementaion suffers from a misunderstanding of the method described in the above article, and that if implemented correctly it could prove more efficient than Oriented Bounding Boxes..
}

\clearpage
\listoffixmes
\tableofcontents

% -*- coding: utf-8 -*-

\section{Introduction}
\label{introduction}

The area of protein folding (and from this the creation of newer and better drugs) with the aid of computers, has since the 60's been an area of much research and experimentation. \Sfixme{Improve this part}

Since proteins cannot physically overlap with another part, it is necessary that fold is tested for overlap before it goes ahead. It is therefore critical that the overlap tests can be done quickly.

One method for detecting collisions is Bounding Volumes Hierarchy (BVH), using a Bounding Volume (BV), where each node in the BVH is enveloped by a BV that covers all the children of the nodes\footnote{The leafs of the node is the actual parts of the protein}. If a BV does not intersect any other BV, then it is clear that the attempted fold is legal, and it can take place. If there is a overlap between two or more BVs, then further checks (lower in the BVH) are needed, to check whether a part of the protein really overlaps another part.

\begin{figure}
\centering
\includegraphics[width=0.5\textwidth]{figures/rss}
\caption{\label{rss-example-figure}An example of a RSS}
\end{figure}

One of possible BV is the Rectangular Swept Sphere (RSS), which is generated by sweeping a sphere (with a positive radius) over a rectangle in 3D space. The resulting volume looks something like a pillow. For an example, see figure \ref{rss-example-figure}. 

I will in this project attempt to implement a reasonably effective RSS BV, with the ability to test for overlap detection, the ability to create a new RSS from a set of points, and the ability to crate a new RSS that contains all the points of 2 (possibly disjoint) RSS'.

\subsection{Scope and Limitations}
\label{scope}
I will not try to implement the RSS in the BVH, and as a consequence not perform tests on it based on active folding of proteins, though I will test it on provided test data derived from snapshots of protein folding.

\subsection{Expectations to the reader}
I expect the reader to have a good grasp of computational geometry and, vector manipulations, and algorithms in general.

\subsection{Terminology}
In this report I will use the term ``tight fit'' to mean that the Bounding Volume around a point-set contains all the points within it, and that it has as little wasted space - i.e. regions of the BV with no points in it, as possible. BV A is a tighter fit than BV B, if both BV's contain all the points, but BV A has a smaller volume than BV B.

\subsection{Guide to this report}
\begin{description}
\item[Section \ref{introduction}] The introduction to the report, which will introduce the subject and explain my goals
\item[Section \ref{rss}] A description of the Rectangular Swept Spheres - how they are defined in theory, and how I have chosen to represent them. This section will also contain information about some of the literature that has worked with RSS.
\item[Section \ref{algorithms}] The algorithms that work on the RSSs, and an analysis of their run time 
\item[Section \ref{implementation}] Notes on the implementation of the algorithms from the previous section and the RSS itself in ProGAL.
\item[Section \ref{results}] A discussion about the results of the implementation together with a comparison with Oriented Bounding Boxes. 
\item[Section \ref{conclusion}] The conclusion of the report, where I will sum up my findings 
\end{description}



\Sfixme{Make introduction conclusion}

% -*- coding: utf-8 -*-

\section{Rectangular Swept Spheres}
\label{rss}

\subsection{Introduction}
In this section I will go into greater detail about the RSS, how it is defined and represented, as well as the literature that I have used in this report.

\subsection{Rectangular Swept Sphere}
The Rectangular Swept Sphere (RSS) is a 3D figure. It is defined as a rectangle in 3D, where a sphere has been swept over its surface - making it resemble a rectangular rounded pillow \Sfixme{is this a reasonable description}. An illustration of a RSS can be found on p. 10 of \cite{Larsen99fastproximity}.\Sfixme{Should I even refer to this} 

\subsection{Representation of RSS}
I have chosen to represent the RSS as a rectangle in 3D together with a radius, as described in \cite{larsen00fast}, \cite{Larsen99fastproximity} and \cite{237244}. The rectangle is made up from a 3d point which is its center, 2 3d unit vectors that describes the orientation \fixme{is this a correct term} of the rectangle, and the length of the of the original 3d vectors. I felt this was best way to represent it, as it adequately described the RSS, took up a minimum of space, contains the vectors, and complies with the representation found in \cite{327244}, which is important since I use the axis-separation test algorithm found in \cite{237244}.


\subsection{Used literature}
For this project I have used the following literature:
\begin{description}
\item[\cite{larsen00fast}] Gives an introduction to some of the problems with making Distance Queries between RSSs. The article refers \cite{Larsen99fastproximity} (by the same authors) which contains more explicit information on how to calculate the distance. 
\item[\cite{Lotan03algorithmand}] Gives a general introduction to Monte Carlo simulation of Proteins, as well as a comparison of several data-structures and algorithms for these, for protein folding \Sfixme{Is it correct? It can be written better}
\item[\cite{Larsen99fastproximity}] A more in-depth description of how the distance queries for the RSSs can be preformed. However, it leaves the most difficult cases to \cite{237244}.
\item[\cite{237244}] A description the algorithm for the most difficult case of RSS distance query. Although the algorithm as described is for Oriented Bounding Boxes (OBB).
\end{description}

% -*- coding: utf-8 -*-

\section{Algorithms}
\label{algorithms}
\subsection{Introduction}
In this section I will go into high-level detail about the principal problems that are to be solved in this project, as well as the algorithms that I have used to solve them. With each algorithm I will include an explanation of their use, an analysis of the algorithms run-time, as well as a discussion of possible alternatives. 

\subsection{Intersection}
In order to be able to use the Rectangular Swept Sphere (RSS) in the [whatever] \Sfixme{find name of framework}, it is paramount that I must find a fast way to detect when 2 RSS' are intersecting, so that it can be decided whether a fold of the protein should go ahead or not. In the following I will describe the methods I have used to check whether or not two RSS' are intersecting.

\subsubsection{Approach}
In the algorithm I have chosen to implement, I will prefer to report a possible intersection where there might not be one, instead of failing to report a possible overlap. While this may give a performance penalty, since the system might have to perform more checks, it is critically important that illegal configurations of the proteins are not accepted. 

\subsubsection{Minimum distance}
In the following let minDist(rec(A), rec(B)) be the minimum distance between the rectangles of RSS A and B, radius(A) is the radius of A.\\

One approach to check if 2 Rectangular Swept Spheres, A and B, overlap, is to check the minimum distance between their rectangles. It is clear that if minDist(rec(A), rec(B)) $<=$ Radius(A) + radius(B) \Sfixme{implement in code and show illustation}, then A and B overlaps. This is the approach found in \cite{Larsen99fastproximity}.

The problem then becomes one of finding the 2 closest points, and calculating the distance.
Given that the rectangles do not overlap, then the possible configurations of the closest points are, according to section 4.2.1 of \cite{Larsen99fastproximity}:
\begin{enumerate}
\item Both of the points lie on an edge
\item One of the points lie in the interior of one of the rectangles, while the other either lies on the other rectangles edge or interior.
\end{enumerate}

\subsection{The points lie on 2 edges}
\label{minimumDistance}
Since we are only interested in the smallest minimum distance between the edges, it is clear that we only have calculate the minimum distance between the edges that are closest to each other. The approach in my implementation is the one described in \cite{larsen00fast} and \cite{Larsen99fastproximity}, where the authors exploit the properties of Voronoi diagrams\footnote{For a formal description of Voronoi diagrams see \cite{compgeom:2008} Chapter 7} in order to quickly decide which pair of edges are possible candidate for the smallest distance. For a more detailed justification of this, see \cite{larsen00fast} section 4.2 and \cite{Larsen99fastproximity} section 4.3.1 - particularly Lemma 1.

Fortunately one does not need to calculate the Voronoi diagram in order to detect which Voronoi cell an edge lies in. Let E be the edge whoose closest edges we wish to find, \textbf{n} be the perpendicular vector to E, then the face defined by -\textbf{n} and a point on E will define the half-plane D. All edges that lies, wholey or partly, lies within D are candiates for the closest edge \cite{larsen00fast} \Sfixme{Check that I do not perform plagerism}. However, this has the minor disadvantage that an edge might lie in multiple half-planes, but even in the worst case this would mean that 8 checks would have had to be made (2 checks for each edge), which clearly is better than 16 checks.\Sfixme{perhaps there should be an image illustrating this - yes there proberly should}

The problem then becomes one of finding the minimum distance between 2 line-segments in 3 dimensions, and then returning the smallest of these values. If no possible candidates are found, then infinity is returned.

If RSS' overlap, the closets point the points with the minimum distance to the  , or the minimum distance between the 2 RSS is greater than the maximum radius, then it becomes necessary to run the separating axis test, which is described below\Sfixme{This can be formulated better}.

\subsection{Separating axis test}
\label{sepAxis}
In order to take care of the cases where 2 rectangles either overlap or where the closest point lies inside the interior of the rectangles, I have to do an Axis-separation check, as described in \cite{237244}. The gist of the Axis-separation test is ``that two disjoint convex polytopes in 3-space can always be separated by a plane which is parallel to a face of either polytope, or parallel to an edge from each polytope'' (\cite{237244}, section 5, page. 8). If one or more Axis-separation shows that there might be an overlap, then the system will report that further test are needed.

The Axis-separation test described in \cite{237244} was designed to work on Oriented Bounding Boxes (OBB's), and takes care of rectangle to rectangle axis test as a degenerate case (\cite{237244}, section 5), which is faster. 

However, since each RSS has a radius, it became a question during implementation whether the system should assume that the two objects it was comparing should be treated OBB's (with a length, a width and a height), or if they were rectangles (with only a length and a width). In the end I choose to implement the axis test to take care of OBB's, and treat the overlap for rectangles as a special case (where the height of the two rectangles are 0).

\subsubsection{Order of Minimum distance and Axis separating test}
\label{minAxisOrder}
A pertinent question might be whatever the order of the tests are optimal, or whether it might be better just to run the axis separation test.

In order to have this discussion make any sense, it is important to understand the purpose and the exiting condition for both methods:

\begin{description}
\item[Minimum Distance:] The minimum distance algorithm tries to find all the minimum distances it can find. 
Local best case: None of the edges lies in the half-planes that indicates the  Voronoi cells\Sfixme{other description?}, and it terminates, having found the minimum distance between no edges. The exiting condition of the algorithm is when it either has checked all Voronoi cells \Sfixme{description again?} and calculated all the minimum distances, or when it finds a minimum distance that is smaller than the combined radius of the two RSS' \Sfixme{implement this in the code}. If the algorithm returns with a positive, then we know that the 2 RSS' intersect, but if it returns with a negative, then we do not know whether the 2 RSS' overlap or not. The algorithm favours the cases where the 2 RSS' might overlap.  

\item[Axis Separating:] The axis separation algorithm tries to find an axis that separates the 2 RSS'. In the best case, the first axis will show that the two RSS' are disjoint, and the algorithm terminates. In the worst case the 2 RSS' are not independent, and all 15 (6 axis tests for 3 faces from each box in the RSS \Sfixme{explain definition of RSS back in appropriate section} + 9 axises that are created by a combination of the 3 face vectors from each box\Sfixme{reformulate!}) test have to be made. The algorithm favours the cases where the 2 RSS' are disjoint.
\end{description}

From this it is clear that the order of the algorithm we choose will be influenced by which cases we believe to be the most common, as well as which cases we think of as the most likely. If we think that the 2 RSS' will often be above each other, or that the RSS' often will be separated, then it will be better to only run the Axis-separation test. If on the other hand we believe that RSS' won't be placed above each other, and that they often intersect, then the minimum distance algorithm, followed by the Axis-separation test is preferred.

Since in the context of folding proteins are interested in being told that a fold is illegal as soon as possible (so that the next fold can be tested), I think that first running the Minimum distance algorithm, and then the Axis-separation test would improve the run-time on average\Sfixme{is it average, or is it something else} compared to if we only ran the Axis-separation test.

\subsection{Heuristic for creating RSS for multiple points}
In order for the RSS to be a useful data-structure, it is clear that, given a set of points, we must be able to construct a RSS that contains all of the points. If this was not the case, then we would be unable to use RSS in a Bounding Volume Hierarchy\Sfixme{reference to text and page}.

\Sfixme{Co-variance or just wing it?}

% -*- coding: utf-8 -*-

\section{Implementation}
\label{implementation}

\subsection{Introduction}
In this section I will give an overview of the implementation details of this projects. This will include choice of programming language, a discussion about the possible algorithms that could have been used \Sfixme{should this part be in the algorithm section, and if so should there be an reordering of the sections?}, extra libraries that I have found useful, as well as an overview of the classes and files that I have added to the framework.

\subsection{Choice of programming language}
Since the goal of this project is to implement the Rectangular Swept Sphere into the [whatever] \fixme{what is the framework called - found out} Java framework, there has never been any doubt that as much as possible of this project should be implemented in Java - only using other languages if there were already existing, and efficient code, that implemented critical features, which were to much work to be implemented.

\subsection{Use of Framework}
Since this project is about implementing RSS in to [whatever framework], I will make heavy use of the features already existing in the framework. This will be done both to save time and avoid code duplication, but also under the assumption that the implemented features will be more efficient than what I myself could implement.

\subsection{Possible choice of algorithms}
\subsubsection{Proximity detection}
In the following let minDist(rec(A), rec(B)) be the minimum distance between the rectangles of RSS A and B, and max(radius(A), radius(B)) be the biggest radius of A and B.\\

One approach to check if 2 Rectangular Swept Spheres, A and B, overlap, is to check the proximity of their rectangles. It is clear that if minDist(rec(A), rec(B)) $<=$ max(Radius(A), radius(B)), then A and B overlaps. This is the approach found in \cite{Larsen99fastproximity}.

The problem then becomes one of finding the 2 closest points, and calculating the distance.
If the rectangles of A and B does not overlap, the closest points, as described in section 4.2.1 of \cite{Larsen9fastproximity}, lies:
\begin{enumerate}
\item on 2 edges
\item in the interior of one of the rectangles
\end{enumerate}

\subsection{Between 2 edges}
Since we are only interested in the smallest minimum distance between the edges, it is clear that we only have calculate the minimum distance between the edges that are closest to each other. The approach in my implementation is the one described \cite{Larsen99fastproximity}, where the authors exploit the properties of Voronoi diagrams\footnote{For a formal description of Voronoi diagrams see \cite{compgeom:2008} Chapter 7} to quickly decide which pair of edges are worth testing. For a more detailed justification of this, see \cite{Larsen99fastproximity} section 4.3.1 - particularly Lemma 1.  

Fortunately one does not need to calculate the Voronoi diagram in order to find whether an edge lies in another edges Voronoi cell. Instead, it can be done by checking which edges lies within the half-plane created by the edge.

In the following, let e be the edge in A whose closest edges we want to find.  
In practice I exploit the Voronoi properties, I create the needed plane P by taking e's vector and a point in e, and check which endpoints of the edges in B lies above\Sfixme{does this make sense? I mean sure code wise, but what is the correct termanoligy?} P. The minimum distance between these edges are then found (which resolves into the problem of finding the minimum distance between 2 line-segments).

This process is repeated for all the edges and the smallest of the minimal distances is returned. If the smallest minimal distance is smaller than the radius of the RSS, then there might be an overlap, and further tests are needed.

If no minimum distance is found, or if parts of an edge is found to be above P, and some of it below P, then it will be necesary to runt he Seperating axis test, which is described below.

\subsection{Separating axis test}
\label{sepAxis}
In order to take care of the cases where the 2 rectangles either overlap or where the closest point lies inside the interior of the rectangles, I have to do an Axis-separation check, as described in \cite{237244}. The gist of the Axis-separation test is ``that two disjoint convex polytopes in 3-space can always be separated by a plane which is parallel to a face of either polytope, or parallel to an edge from each polytope'' (\cite{237244}, section 5, page. 8). If one or more Axis-seperation shows that there might be an overlap, then the system will report that further test are needed.

The Axis-separation test described in \cite{237244} was designed to work on Oriented Bounding Boxes (OBB's), and takes care of rectangle to rectangle axis test as a degenerate case (\cite{237244}, section 5), which is faster. 

However, since each RSS has a radius, it became a question during implementation whether the system should assume that the two objects it was comparing should be treated OBB's (with a length, a width and a height), or if they were rectangles (with only a length and a width). In the end I choose to implement the axis test to take care of OBB's, and treat the overlap for rectangles as a special case (where the height of the two rectangles are 0).

\subsection{Heuristic for creating RSS for multiple points}

% -*- coding: utf-8 -*-

\section{Results}
\label{results}
\subsection{Introduction}
In this section I will discuss the results of the project. I will start by discussing the speed of the implemented features, and then talk about volume.

\subsection{Implemented features}
I have successfully implemented the RSS BV as well as its dependent class, a rectangle in 3D. 
All of the algorithms described in the last section has been implemented.
Furthermore I have implemented all the required methods for the 2 classes.

\subsection{Test data}
For the test results I have used 2 data sets. The randomly generated data, and a sample of real world proteins, supplied to me by Rasmus Fonesca.

The random test data will be used as a general stress test of the implemented algorithm (since I will be able to generate a large number of point sets), while the real world data will be used to gauge its usefulness in practice.

\subsubsection{Randomly generated test data}
For the randomly generated test data, has been made by constructing 2 arrays of 1000 point sets with 25 points in each set. The points have all been randomly created by using Java's built in Random function and Point3d's own Random constructor.

I maintain 2 arrays for each type of BV I create, with the intention of checking each element from the first array against all elements in the second array. This is done in order to ensure that in practice a BV is never checked against itself, and also so that I am quickly able to create large number of tests, without having to use too much system memory. \\ 

For each point set in each array, I have created a 3 new BV's, one for
each of the types OBB, LSS, and RSS. Each BV contains the all points
in the point set it was created from. Thus I create 2000 (split into 2 arrays of 1000 each)
new BV of each type. \\ 

\subsubsection{Real world data}
The real world data has been generated by parsing a text file containing definitions for real world proteins. Each Protein contains 10 chunk-chains, each chunk-chain is defined by a set of points. For each chunk-chain point set I will create a BV, which I will test for overlap against all other chunk-chains in the same protein (since this sub test is to check its ability to handle real world data, it would make little sense to test it against the chunk-chains from other proteins). 
I will perform this test for OBBs, LSS' and RSS'

\subsection{Data extracted from tests}

I will compare the efficiency of the tests based on the metrics of BV volumes, and the wall clock time required to check whether 2 BV's overlaps or not. I feel having both volume and wall clock time is necessary, since it might give hints whether certain BV properties can mean measurable time savings, despite being a worse fit. 

I perform 2 test: 
\begin{description}
\item[Overlap:] I test the wall clock time required to perform the overlap test,
  for all possible BV combinations, for all 3 types of BV. The creation of the BV's are as described above.
 
\item[Volume:] I find the average volume of the individual BV's that cover a single point set. For the randomly generated data I will also generate the average of all possible BV combinations that can be created from combining 2 BV's. I have not done this for the real world data, as the concept of average for combined BV's makes no sense, since BV's will only be combined within their own protein.
\end{description}

All tests have been executed on a computer running Windows XP Pro (Service pack 3) with a 8x3.06Ghz Core i7-950 and has 6 GB of RAM.s

\subsection{Overlap detection}

\subsubsection{Randomly generated data}

\begin{table}
\begin{tabular}{c|c|c|c|c}\\ 
& RSS: & RSS no-min & OBB: & LSS:\\ 
Total time spent: & 0,7660 & 0,6880 & 0,7030 & 0,1090\\ 
Number of overlaps: &1000000 & 1000000 & 1000000 & 1000000\\ 
\end{tabular}
\caption{\label{overlap-table}The table of the time used for the
  different overlaps checks. All of the times are in wall clock time seconds. The
  check reading ``RSS no-min'' is a RSS overlap check that only runs the axis separation test, and no minimum distance check}
\end{table}

As it is clear from table \ref{overlap-table}, the current RSS overlap algorithm is slower than both the one for OBB's, and LSS'. However, seeing that the RSS both has a much larger average volume than the other 2, could indicate that the properties of the RSS either makes up for its greater volume. It is very interesting that the RSS no-min uses slightly less time than OBB test, despite the fact that they both use the axis-separation method.

It is worth noting that for the RSS, the majority of the used time comes from the Axis-separation check (since it is clear that RSS no-min takes over half the time a regular RSS test takes).

\subsubsection{Volume}
\begin{table}
\begin{tabular}{c|c|c}\\ 
RSS average & OBB average & LSS average\\ 
10,200 & 1,054 & 1,270\\ 
\hline 
Combined RSS average & Combined OBB average & Combined LSS average\\ 
11,333 & 3,901 & 1,569\\ 
\end{tabular}
\caption{\label{volume-table} The average volume needed by the
  different BV to contain the points. The first row of values are the average
  volumes for each of the 2000 different BV that are produced, while
  the second row of values are the average volumes of the 1.000.000
  different combinations of BV's.}
\end{table}

From table \ref{volume-table} it is clear that the current implementation of the RSS is worse than both OBBs and LSS', both combined and on average.

\subsection{Real world data}
\label{realWorldData}
\begin{table}
\begin{tabular}{c|c|c|c|c}\\ 
& RSS: & RSS no-min & OBB: & LSS:\\ 
Total time spent: & 0,050 & 0,050 & 0,009 & 0,005\\ 
Number of overlaps: &16011 & 16011 & 6783 & 2431\\ 
\end{tabular}
\caption{\label{overlap-realtable}The table of the time used for the
  different overlaps checks. All of the times are in wall clock time seconds. The
  check reading ``RSS no-min'' is a RSS overlap check that only runs the axis separation test, and no minimum distance check}
\end{table}

It is clear from table \ref{overlap-realtable}, that the current implementation of RSS overlap algorithm is much slower than the ones for  OBBs, and  LSS'. Furthermore it is clear that in this case the main slowdown comes from the minimum distance check, and not the Axis-separation, since the No-min tests for the RSS perform just as fast as the one for the OBB, despite having a greater volume. It is furthermore interesting to note that the overlap check that has the minimum distance check reports slightly more overlaps, than the test which does not. Since the RSS no-min tests uses the Axis-separation method which always detects an overlap, this indicates that the current implementation of the minimum distance algorithm reports overlaps which does not exits. However, the extra number of possible overlaps detected might not prove a too significant a performance hit in practice.

It is worth noting, that unlike the last check, the main time expenditure for the program lies with the minimum distance check (Since the RSS no-min test clearly takes under half the time of the normal RSS overlap test).

\subsubsection{Volume}
\begin{table}
\begin{tabular}{c|c|c}\\ 
RSS average & OBB average & LSS average\\ 
16130,036 & 1022,997 & 1,206\\ 
\end{tabular}
\caption{\label{volume-realtable} The average volume needed by the
  different BV to contain the points.}
\end{table}

Interestingly table \ref{volume-realtable} gives a far worse average for the RSS BV, compared to the other 2 BV's than from the randomly generated data. This would indicate that it would be worth looking into reducing the volume.

\subsection{Conclusion}
I have in this section shown that the current implementation of the RSS performs worse than both the OBB and the LSS on both metrics of volume and wall clock time when performing overlap detection.

The differences in proportions of time spent on Minimum distance and Axis-separation for the RSS', for the 2 types of test data, can most likely be attributed to the fact that with the randomly generated data, every single BV overlaps (see table \ref{volume-table}), while with the real world data (see table \ref{volume-realtable}), far fewer does. As I explained in section \ref{sepAxis}, the Axis-separation algorithm terminates as soon it finds a axis that separates the 2 BV's, but in the case where none exists, it will do all 15 tests. Since we have established that all BV's for the randomly generated data overlaps, it is clear that all 15 checks for the RSS must be performed.  

If the RSS BV is to be used in practice, these limitations must be overcome. There is furthermore a clear indication that for real world data, the bottleneck of the current implementation lies in the Minimum distance algorithm, which should be optimized.

% -*- coding: utf-8 -*-

\section{Conclusion}
\label{conclusion}
\subsection{Introduction}
In this section I will sum up the project and discuss possible future projects. 

\subsection{Efficiency of solution}
It is clear that compared to other, already implemented Bounding Volumes in ProGAL, such as LSS' (Capsule3d), that the the detection speed of overlapping RSS' leaves much to be desired, as does its volume.

\subsection{Future work}
In order to properly gauge the efficiency of my implementation, I feel that it should be implemented in the Bounding Volume Hierarchi, and thoroughly tested against Oriented Bounding Boxes. I furthermore think that both the overlap code, and the tightness of the volume should be optimized, if the RSS is ever to be considered as a data-structure for protein folding. 

Another Swept Sphere variations worth looking into is Triangular Swept Spheres (TSS), where instead of a point, line or rectangle (as is the case in PSS, LSS, and RSS respectively), the midpoint of the sphere swept through a triangle. 

Another project could be whether a hybrid OBB/RSS data-structure would be interesting (such as giving the OBB a radius that only works in the direction of the normal vector of the plane) and try to implement different overlay detection methods (such as implementing a OBB version of the minimum distance algorithm implemented for the RSS - where only the edges that are in the direction of the normal edges are checked).


\bibliography{RSSbib}{}
\bibliographystyle{plain}
\end{document}
