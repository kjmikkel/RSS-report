% -*- coding: utf-8 -*-

\section{Introduction}
\label{introduction}
The influence of computers in modern medicine is becoming increasingly important. In order to find new and better drugs, or gain a greater understanding of diseases, scientist have turned to computer in order to simulate various parts of the human biology.

One often used method is that of protein folding. Since parts of a protein cannot physically overlap with another part, the program has to check whether there is a collision or not. Since this test has to be done often, it is critical that the collision test can be done cheaply. 
One method for detecting collisions is Bounding Volumes (BV), using Bounding Volume Hierarchy (BVH), where each node in the BVH is enveloped by a BV that covers the all children of the nodes (with the leafs of the node being the actually parts of the protein itself). If a BV does not intersect any other BV, then it is clear that the attempted fold is legal, and it can take place. If there is a collision between two or more BVs, then further checks are needed, in order to check whether there truly is a collision.

One of possible BV is the Rectangular Swept Sphere (RSS), which is generated by sweeping a sphere (with a non-zero radius) all over a rectangle in 3D space. The resulting volume should look something like a pillow. For an illustration see figure 1 in \cite{Larsen99fastproximity}.

I will in this project attempt to implement a reasonably effective Rectangular Swept Sphere (RSS) BV, complete with overlap detection of other RSSs and an algorithm to make an BV for multiple RSSs.

\subsection{Scope and Limitations}
\label{scope}

\subsection{Expectations to the reader}
I expect that the reader has a rudimentary knowledge of the problems encountered with protein folding, as well as familiarity with the concepts such as Chain Trees, Bounding Volumes and Swept Spheres. Furthermore I expect the reader to have a good grasp of computational geometry and, vector manipulations and algorithms in general.\Sfixme{are more expectations needed?}

\subsection{Learning targets and objectives}
\label{learning}
\subsubsection{Learning targets}
After having completed this project I will have learned:
\begin{itemize}
\item the basics of the RSS and which problem there exists with implementing it in practice.
\item the basics of  Bounding Volumes (BV), and the problems there exists when trying to use them for simulate protein-folding.
\item to identifying the possible problems there exists when implementing geometrical data structures and their algorithms in 3D
\end{itemize}

\subsection{Objectives}
\begin{itemize}
\item Implementation of a data-structure that represents the Rectangular Swept Sphere (RSS) in Java.  
\item Detection of overlap between RSSs.
\item Implementation of an algorithm that can make a RSS of multiple RSSs.
\item If there is sufficient time I will try to implement a heuristic to that can make a RSS from multiple points.
\item If there is further time I will attempt to compare RSS and LSS.
\end{itemize}

\subsection{Terminology}

\subsection{Guide to this report}
\begin{description}
\item[Section \ref{introduction}] The introduction to the report, which will introduce the subject and explain my goals
\item[Section \ref{rss}] A description of the Rectangular Swept Spheres - how they are defined in theory, and how I have chosen to represent them. This section will also contain information about some of the literature that has worked with RSS. \Sfixme{Should the discussion about the literature be here, or somewhere else?}
\item[Section \ref{algorithms}] The algorithms that work on the RSSs, and an analysis of their space requirements and run time \Sfixme{Do i need to mention how much time they use?}
\item[Section \ref{implementation}] Notes on the implementation of the algorithms from the previous section and the RSS itself in [java framework written by Pawel and Rasmus] \Sfixme{Get name from Pawel and Rasmus' framework}
\Sfixme{Find out whether this part is to be included or not}
%\item[Section \ref{comparison}] A comparison of RSS compared to ISS and other Bounding Volumes - only do this if we have the time 
\item[Section \ref{conclusion}] The conclusion of the report, where I will sum up the findings  
\Sfixme{Is more needed? Is there anything I have left out?}
\end{description}
