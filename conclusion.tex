% -*- coding: utf-8 -*-

\section{Conclusion}
\label{conclusion}
\subsection{Introduction}
In this section I will attempt 

Finally I will discuss possible future projects.

\subsection{Efficiency of solution}
While it at the current state is hard to truly gauge the quality of the implementation, it is clear that compared to other, already implemented Bounding Volumes in ProGAL, such as Capsules, that the the detection speed of overlapping RSS' leaves much to be desired.

\subsection{Future work}
In order to properly gauge the efficiency of my implementation, I feel that it should be implemented with Bounding Volume Hierarchies, and thoroughly tested against Oriented Bounding Boxes. I am however of the opinion that the current implementation of the RSS', and especially the code to detect overlap, leaves room for optimizations, which also should be part of the project. Another aspect worth investigating would be the implementation of non-axis aligned RSS (as mentioned in \ref{RSS3d}), and compare it with the regular RSS.

Another Swept Sphere variation it might be worth looking into is Triangular Swept Spheres (TSS), where instead of a rectangle or a line (as is the case in RSS and LSS) a triangle is used as the surface for the sphere to swept through.
