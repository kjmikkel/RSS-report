% -*- coding: utf-8 -*-

\section{Conclusion}
\label{conclusion}
\subsection{Introduction}
In this section I will sum up the project and discuss possible future projects. 

\subsection{Efficiency of solution}
It is clear that compared to other, already implemented Bounding Volumes in ProGAL, such as LSS' (Capsule3d), that the the detection speed of overlapping RSS' leaves much to be desired, as does its volume.

\subsection{Future work}
In order to properly gauge the efficiency of my implementation, I feel that it should be implemented in the Bounding Volume Hierarchi, and thoroughly tested against Oriented Bounding Boxes. I furthermore think that both the overlap code, and the tightness of the volume should be optimized, if the RSS is ever to be considered as a data-structure for protein folding. 

Another Swept Sphere variations worth looking into is Triangular Swept Spheres (TSS), where instead of a point, line or rectangle (as is the case in PSS, LSS, and RSS respectively), the midpoint of the sphere swept through a triangle. 

Another project could be whether a hybrid OBB/RSS data-structure would be interesting (such as giving the OBB a radius that only works in the direction of the normal vector of the plane) and try to implement different overlay detection methods (such as implementing a OBB version of the minimum distance algorithm implemented for the RSS - where only the edges that are in the direction of the normal edges are checked).
