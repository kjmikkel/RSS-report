% -*- coding: utf-8 -*-

\section{Implementation}
\label{implementation}

\subsection{Introduction}
In this section I will give an overview of the implementation details of this project. This will include choice of programming language, libraries that I have found useful, as well as an overview of the classes and files that I have added to the framework.

\subsection{Choice of programming language}
Since the goal of this project is to implement the Rectangular Swept Sphere into the ProGAL Java framework, there has never been any doubt that as much as possible of this project should be implemented in Java - only using other languages if there was already existing, efficient, code that implemented critical features. This has not been the case.

\subsection{Use of Framework}
\label{framework}
Since this project is about implementing RSS in the ProGAL framework, I will make heavy use of the features already existing in the framework. I have done so in order to save time, and avoid code duplication, and because already implemented features are likely to be more efficient than what I myself could implement.

\subsection{Classes implemented}
In this project I have implemented 2 classes:
\begin{itemize}
\item Rectangle3d
\item RSS3d
\end{itemize}

which primarily uses these custom classes form ProGAL:

\begin{itemize}
\item[Point3d:] A point in 3D
\item[Vector3d:] A vector in 3D
\item[Volume3d:] An interface that requires the implementation of an \texttt{overlaps} and \texttt{volume}, which respectively tests whether 2 Volume3d overlaps, and returns the volume of the Volume3d object as a double.
\end{itemize}

\subsubsection{Rectangle3d}
\label{rectangle3d}
The Rectangle3d class is used to represent a rectangle in 3 dimensional space. The rectangle itself is represented by a center point (Point3d), 2 unit vectors (Vector3d) which describe the plane that the rectangle lies in, and 2 doubles that contains the length of the vectors. To make it easier to implement in the Bounding Volume Hierarchy, rectangle3d class implements the interface Volume3d.
It is in this class that I have implemented the minimum distance method, which I justify by the fact that it is between 2 rectangles, and not specificly between 2 RSS'

\subsubsection{RSS3d}
\label{RSS3d}
The RSS3d class is used to represent a Rectangular Swept Sphere in 3 dimensional space. The RSS is represented by a radius which is given as a double, and the rectangle is a Rectangle3d, as descried above, and a Box3d that contains the entire RSS. In order to comply with the other structures, the rectangle3d class implements the interface Volume3d.

In order to check whether 2 RSS' overlap, one of the RSS' overlap method is called with the other RSS as argument. 

\subsection{Axis separation test}
\label{implementation_axis_sep}
Since I have to generate an OBB (in practice an instance of the class Box3d) for the RSS, I have opted to use Box3d's built in Axis separation test, instead of writing my own, for the reasons described in \ref{framework}.
